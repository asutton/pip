\documentclass{article}
\usepackage{parskip}

\begin{document}
Conventional switches, or layer 2 devices in the OSI model, are configured using vendor-specific interfaces that lack scaleability and flexibility. In addition, manual switch configuration is heavily prone to error and compromise. Software-defined networking (SDN) is a paradigm that helps to mitigate user error in packet switching and increase network scaleability by allowing switches to be programmed through a high-level programming language. However, there is no standard language for SDN, some programmers have tried to use C or Python, others have developed domain-specific languages to achieve these purposes.

The problem with typical programming languages is that the semantics of these programming languages are informally designed, often rife with side effects, and perhaps most pertinent for packet switching: prone to buffer overflows. We have presented a new domain specific programming language and virtual machine to model an OpenFlow-style network, with a formally-defined type system and structural operational semantics, thus allowing properties to be proven about software-defined networking programs.
\end{document}
