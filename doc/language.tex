\section{The Pip Language}
This chapter (appendix?) will serve as a reference for the grammar and syntax of the Pip programming language.

Pip is an S-expression-based language. There are 6 basic S-expressions used in the Pip language:
\begin{enumerate}
  \item Pip
  \item Table
  \item Action List
  \item Action
  \item Rule List
  \item Rule
\end{enumerate}

\subsection{Pip S-Expression}
Pip S-Expressions are the expressions that contain the program itself, comparable to the main function in C. Pip expressions can contain any amount of table expressions as parameters. Example:
\begin{mdframed}
\begin{verbatim}
(pip (table ...) (table...))
\end{verbatim}
\end{mdframed}

\subsection{Table S-Expression}
Table S-Expressions define a match table. They contain, in order, a name, a table type (exact match tables are currently the only supported table type), a preparation action list (an action list that sets the key register of the table) ending with a $\mathct{match}$ action, and a rule list that defines the matching rules of the table. For example:
\begin{mdframed}
\begin{verbatim}
(table table_name exact
  (actions
    (set (bits key 0 32) (int i32 0))
    (match)
  )
  (rules
    (rule ...)
  )
)
\end{verbatim}
\end{mdframed}
In this table, an exact-match table called \textit{table\_name} is declared. Its key register is $\mathct{set}$ to the 32-bit integer 0 and matching begins. The rules in the rule list will then be examined and possibly executed. Rules and key registers will be explained later in the chapter.

\subsection{Action Lists and Actions}
Actions are the basic instructions in Pip. Actions Lists are simply sequences of one or more actions. Pip has 9 actions, described as follows: 
\begin{enumerate}
  \item $\mathct{write}$ writes an action the stored action list of the virtual machine. 
  \item $\mathct{clear}$ clears all actions in the stored action list.
  \item $\mathct{drop}$ drops the packet and discontinues processing.
  \item $\mathct{match}$ ends the prep list of a table and begins matching.
  \item $\mathct{goto}$ jumps to another table. Only tables that appear later in the program may be jumped to.
  \item $\mathct{output}$ sends the packet out a specific port.
  \item $\mathct{advance}$ increments the context variable $\mathct{header\_offset}$ by the specified amount. This determines the distance in bits of the $\mathct{header}$ address space from the $\mathct{packet}$ address space.
  \item $\mathct{copy}$ copies n bits from one bitfield to another. The programmer may not copy past th eend of an address space.
  \item $\mathct{set}$ sets a bitfield to some literal value.
\end{enumerate}
An example action list follows:
\begin{mdframed}
\begin{verbatim}
(actions
  (set (named_field tcp.dst) 443) (drop))
\end{verbatim}
\end{mdframed}
An action is marked by the $\mathct{action}$ keyword. Individual actions are not marked by a keyword as they only appear within action lists and write action parameters.

\subsection{Rule Lists and Rules}
Rules are the entries in a table that are matched upon. A rule has a key (not to be confused with a $\mathct{key}$ register) that is equality compared to the $\mathct{key}$ register. In an exact match table, a rule key is of type $\mathct{int}(n)$. Along with a key, a rule contains an action list that will be executed if the rule key matches the key register. A rule list is a sequence of one or more rules. An example rule list follows:
\begin{mdframed}
\begin{verbatim}
(rules
  (rule (int i32)
    (actions ...)))
\end{verbatim}
\end{mdframed}
A rule is marked by the $\mathct{rule}$ keyword and a rule list by the $\mathct{rules}$ keyword.

\subsection{Key Register}
The $\mathct{key}$ register is a 64-bit field that is contained in the table data structure. The $\mathct{key}$ register is equality compared to the table's $\mathct{rule}$ keys. In other words, matching in a table works similarly to switch statements in C.

\subsection{Address Spaces}
There are 4 address spaces in the Pip language and virtual machine:
\begin{enumerate}
  \item $\mathct{packet}$ denotes the 0th bit in the packet.
  \item $\mathct{header}$ denotes the 0th bit in the packet plus the value of the $\mathct{header\_offset}$ context variable.
  \item $\mathct{meta}$ is the 64-bit metadata register. It can be used to store values for scratch processing.
  \item $\mathct{key}$ is the 64-bit key register.
\end{enumerate}

\subsection{Reserved Ports}
Pip inherits the following reserved ports from the OpenFlow standard:
\begin{enumerate}
  \item $\mathct{drop}$ or port 0. Outputting to this port has the same effect as the $\mathct{drop}$ action.
  \item $\mathct{all}$ outputs to all ports on the device.
  \item $\mathct{controller}$ sends the packet to some non-Pip program. Our implementation uses C++ programs.
  \item $\mathct{in\_port}$ outputs to the physical ingress port of the packet.
  \item $\mathct{any}$ outputs to a random port.
  \item $\mathct{local}$ outputs to the local IP stack.
\end{enumerate}

\subsection{Named Fields}
Named fields are bitfields; they have type $\mathct{bits}$. They are equivalent in every way to a bitfield in the $\mathct{packet}$ address space, and only exist as syntactic sugar, but help to prevent memory access violations. A named field should always be preferred to a raw bitfield if available. Named fields exist in Pip for Ethernet frames, IPv4 fields, and TCP fields. The naming scheme uses common abbreviations, for example $\mathct{eth.dst}$ representing the Ethernet destination MAC address, and $\mathct{ipv4.len}$ representing the IPv4 header length. IPv6 and UDP will be added in the future.
%% \begin{enumerate}
%%   \item $\mathct{eth.dst}$ 
%%   \item $\mathct{eth.src}$
%%   \item $\mathct{eth.type}$
%%   \item $\mathct{ipv4.vhl}$
%%   \item $\mathct{ipv4.tos}$
%%   \item $\mathct{ipv4.len}$
%%   \item $\mathct{ipv4.id}$
%%   \item $\mathct{ipv4.frag\_offset}$
%%   \item $\mathct{ipv4.ttl}$
%%   \item $\mathct{ipv4.protocol}$
%%   \item $\mathct{ipv4.checksum}$
%%   \item $\mathct{ipv4.src}$
%%   \item $\mathct{ipv4.dst}$
%%   \item $\mathct{tcp.src}$
%%   \item $\mathct{tcp.dst}$
%%   \item $\mathct{tcp.seq}$
%%   \item $\mathct{tcp.ack}$
%%   \item $\mathct{tcp.offset}$
%%   \item $\mathct{tcp.flags}$
%%   \item $\mathct{tcp.window}$
%%   \item $\mathct{tcp.checksum}$
%%   \item $\mathct{tcp.urgent\_ptr}$
%% \end{enumerate}
