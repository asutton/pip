
\section{Literature Survey}
1) What is the problem they try to solve? \\
2) How are they solving it? \\
3) What is the difference between their research and ours?  (What they solve vs.
what we solve) \\


\noindent \emph{Languages for Software-Defined Networks}

The Frenetic project aimed to raise the level of abstraction for 
programming SDNs \cite{Foster2013}. Frenetic provides a suite of declarative
abstractions 
for querying network state, defining forwarding policies, and updating 
policies in a consistent way. This allows the project to replace the 
available low-level imperative interfaces. This was attained by designing 
the constructs in a modular way and implementing them by compiling
them down to low-level OpenFlow forwarding rules. \\

\noindent \emph{Eliminating Network Protocol Vulnerabilities Through Abstraction
	and Systems Language Design}

In a paper written by Casey, a systems programming language that allows
programmers to capture network protocol message structure and constraints was
introduced \cite{Casey2013}. In that paper, safe and efficient implementations
of standard message handling operations were synthesized by a compiler, and
whole-program analysis was used to ensure constraints were never violated. It
was shown that these categories for message constraints are responsible for known
vulnerabilities, and using their tools they showed that even some live Internet traffic 
violates these constraints.
This allows for full program analysis, providing stronger safety guarantees and offering 
domain 
specific optimization that is exceedingly difficult to accomplish by hand or impossible
with a DSL. \\ 

\noindent \emph{Design Principles for Packet Parsers}
In a paper written by Gibb, the trade-offs in parser design were described, design
principles for switch and router designers were identified, and a parser generator that
outputs synthesizable Verilog that is available for download was described
\cite{Gibb2013}. The
contributions of this paper include introducing the problem of designing streaming packet
parsers, outlining the similarities and differences to instruction decoding in processors,
describing a new methodology for designing streaming parsers for several optimizations, a
and demonstrated that modern parsers occupy approximately 1-2\%. \\

\noindent \emph{Extending Networking into the Virtualization Layer}

A network switch built specifically for virtual environments, Open vSwitch,
exports an external interface
for fine-grained control of configuration state and forwarding behavior \cite{Pfaff2009}.
Open vSwitch provides
connectivity between virtual machines and physical interfaces. It exports
interfaces for manipulating
the forwarding state and managing configuration state at runtime  
\\

\noindent \emph{Frenetic: A Network Programming Language} \\

\noindent \emph{Leaping multiple headers in a single bound: wire-speed parsing using the Kangaroo system} \\


\noindent \emph{Modular SDN Programming with Pyretic} \\

\noindent \emph{PFPSim: A Programmable Forwarding Plane Simulator} \\




