
\section{Literature Survey}
1) What is the problem they try to solve?
2) How are they solving it?
3) What is the difference between their research and ours?  (What they solve vs. what we solve)

\textbf{Languages for Software-Defined Networks} 

The Frenetic project aimed to raise the level of abstraction for 
programming SDNs \cite{Foster2013}. Frenetic provides a suite of declarative abstractions 
for querying network state, defining forwarding policies, and updating 
policies in a consistent way. This allows the project to replace the 
available low-level imperative interfaces. This was attained by designing 
the constructs in a modular way and implementing them by compiling
them down to low-level OpenFlow forwarding rules. \\

\noindent \textbf{Eliminating Network Protocol Vulnerabilities Through Abstraction and Systems Language Design
}

In a paper written by Casey, a systems programming language with abstractions capturing network
protocol vulnerability constraints was introduced. \cite{Casey2013}. This programming language
allows programmers to capture network protocol message structure and constraints. 


\noindent \textbf{Design Principles for Packet Parsers}


\cite{Gibb2013} \\

\noindent \textbf{Extending Networking into the Virtualization Layer}

A network switch built specifically for virtual environments, Open vSwitch, exports an external interface
for fine-grained control of configuration state and forwarding behavior. Open vSwitch provides
connectivity between virtual machines and physical interfaces. It exports interfaces for manipulating
the forwarding state and managing configuration state at runtime  
\cite{Pfaff2009} \\

