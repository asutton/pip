\documentclass{article}
\usepackage{mathpartir}

\begin{document}
\section{Type System}
  \subsection{Types}
  it\newline
  range\newline
  location\newline
  port\newline
  unit\newline
  \subsection{Inference Rules}
  \subsection*{Expressions}
  \begin{mathpar}
    \inferrule* [right=\quad int \qquad (1)]
               {\\}
               {\textbf{int it} \ (decimal-literal) : it}
    \newline \newline
    \inferrule* [right=\quad range \qquad (2)]
               {\Gamma \vdash e_1:it \\ \Gamma \vdash e_2:it}
               {\Gamma \vdash \textbf{range}\  e_1 \  e_2 : it}
    \newline \newline
    \inferrule* [right=\quad port \qquad (3)]
               {\Gamma \vdash \textbf{port} \ e_1 \\ \Gamma \vdash e_1:it}
               {\Gamma \vdash \textbf{port} \ e_1 : port}
    \newline \newline
    \inferrule* [right=\quad offset \qquad (4)]
               {\Gamma \vdash pos : it \\ \Gamma \vdash len : it}
               {\Gamma \vdash \textbf{offset} \  (as) \  pos \  len : location}
    \newline \newline
    \inferrule* [right=\quad field \qquad (5)]
               {\Gamma(\textit{identifier}) = (l, \mu) \\ l \rightarrow location}
               {\Gamma \vdash \textit{identifier} : location}
    \newline \newline
    \inferrule* [right=\quad miss \qquad (6)]
               {\Gamma (\textit{identifier}) = (l, \mu) \\ l \rightarrow table}
               {\Gamma \vdash \textbf{miss} : type(key(l))}
    \newline \newline
    \inferrule* [right=\quad ref \qquad (7)]
               {\Gamma (\textit{identifier}) = (l, \mu) \\ l \rightarrow table}
               {\Gamma \vdash \textit{identifier} : unit}
    \newline \newline

  \end{mathpar}
      \textit{Comments:} \newline
    (6) The miss expression will have the same type as the key of the table its rule is defined within.
    \newline
  \subsection*{Actions}
  \begin{mathpar}
    \inferrule* [right=\quad write \qquad (8)]
                {\\}
                {\Gamma \vdash \textbf{write} \ (action) : unit}
    \newline \newline
    \inferrule* [right=\quad clear \qquad (9)]
                {\\}
                {\Gamma \vdash \textbf{clear} : unit}
    \newline \newline
    \inferrule* [right=\quad drop \qquad (10)]
                {\\}
                {\Gamma \vdash \textbf{drop} : unit}
    \newline \newline
    \inferrule* [right=\quad match \qquad (11)]
                {\\}
                {\Gamma \vdash \textbf{match} : unit}
    \newline \newline
    \inferrule* [right=\quad goto \qquad (12)]
                {\Gamma \vdash e_1 : it}
                {\Gamma \vdash \textbf{goto} \ e_1 : unit}
    \newline \newline
    \inferrule* [right=\quad output \qquad (13)]
                {\Gamma \vdash e_1 : port}
                {\Gamma \vdash \textbf{output} \ e_1 : unit}
    \newline \newline
    \inferrule* [right=\quad output \qquad (14)]
                {\Gamma \vdash e_1 : port}
                {\Gamma \vdash \textbf{output} \ e_1 : unit}
    \newline \newline
    \inferrule* [right=\quad advance \qquad (15)]
                {\Gamma \vdash e_1 : it}
                {\Gamma \vdash \textbf{advance} \ e_1 : unit}
    \newline \newline
    \inferrule* [right=\quad copy \qquad (16)]
                {\Gamma \vdash e_1 : location \\ e_2 : location \\ e_3 : it}
                {\Gamma \vdash \textbf{copy} \ e_1 \ e_2 \ e_3 : unit}
    \newline \newline
    \inferrule* [right=\quad set \qquad (16)]
                {\Gamma \vdash e_1 : location \\ e_2 : it \\ e_3 : it}
                {\Gamma \vdash \textbf{set} \ e_1 \ e_2 \ e_3 : unit}
    \newline \newline
  \end{mathpar}
  \textit{Comments:} \newline
  (16) e2 does not need to be type it. It can be any value type. TODO: create a system that represents values which have precision.
\end{document}
