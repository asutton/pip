
\section{Abstract syntax}
In this section, we explain the abstract syntax of the Pip programming
language. 
Pip uses a type system to ensure semi-semantically correct instructions. Instructions, known as actions, can take well-formed expressions or actions as parameters. See Section 4.3 for specification of action parameters. This section defines abstract syntax in Backus-Naur Form, an interested reader can learn more about this notation in \Cite{Aho2007} and \Cite{Pierce2000}.

\subsection{Types}

\begin{align*}
\tau ::=~& \mathct{int}(n)        & \text{integer types} \\
         & \mathct{bits}(a, o, l) & \text{bitfields} \\
         & \mathct{port}          & \text{ports} \\
         & \mathct{table}         & \text{tables} \\
\end{align*}

\begin{enumerate}
\item $\mathct{int}(n)$ types are unsigned integers with a bit-width of size \textit{n}.
\item $\mathct{bits}$ types are bitfields within the storage space of the virtual machine. Bitfields are defined as having an \textit{address space} that denote which area of storage they reside: packet, header, key, meta, ingress\_port, and physical\_port. A bitfield must also specify its length in bits as well as its offset from bit 0 of the address space, again in bits.
\item $\mathct{port}$ types represent numbered ports. Some reserved ports are named, and occupy a static number. For example, the $\mathct{drop}$ port is assigned to 0.
\item $\mathct{table}$ types are data structures that contain all other Pip constructs. A table has a \textit{key register} to be matched against. The programmer defines the value of the key register, which is computed at runtime. Tables have \textit{rules} which, if equal to the key register, will execute a block of actions.
\end{enumerate}

\subsection{Expressions}
\begin{align*}
  \text{expressions} ::=~& \mathct{int}(w) n    & \text{where} \ w \ \text{and} \ n \ \text{are integer literals.}\\
  & | \ \mathct{port} \ n   & \text{where } n \text{ is an integer literal.}\\
  & | \ \mathct{bits} \ as \ pos \ len &\text{where } pos \text{ and } len \text{ are integer literals.} \\ & &\text{and } as \text{ is an address space.}\\
  & | \ \mathct{miss}\\
  & | \ \mathct{ref} \ id    &\text{where } id \text{ is the name of a table.}
\end{align*}
\subsection{Actions}

Actions are instructions that either have no parameters, an expression as a parameter, or another action as a parameter.
\begin{align*}
actions::=~&  \mathct{write} \ action     & action \text{ must be an } \mathct{action}.\\
  & | \ \mathct{clear}\\
  & | \ \mathct{drop}\\
  & | \ \mathct{match}\\
  & | \ \mathct{goto} \ table             & \text{table must be of type } \mathct{table}.\\
  & | \ \mathct{output} \ port            & \text{port must be of type } \mathct{port}.\\
  & | \ \mathct{advance} \ n              & \text{n must be an integer literal.}\\
  & | \ \mathct{copy} src \ dst \ n       & src \text{ and } dst \text{ must be of type } \mathct{bits} \\
                                         && \text{n must be of type } \mathct{int}(w).\\
  & | \ \mathct{set} field \ value        & field \text{ must be of type } \mathct{bits} \text{ and }\\
                                         && value \text{ must be of type } \mathct{int}(w).
\end{align*}
