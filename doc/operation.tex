
\section{Operational Semantics}

\subsection{Notation}

The state of the evaluator, $\sigma$, is the tuple consisting of \\
packet, $\pi$\\
action\_list, $\alpha$ \\
stored\_action\_list, $\Lambda$ \\
context, $\Gamma$, which is the tuple: \\
$\langle ingress\_port, \ egress\_port, \ physical\_port, \ key, \ meta, \ header\_offset, \ table \rangle$ \\ \\
Compactly, $\sigma = \langle \pi, \alpha, \Lambda, \Gamma \rangle$
\\ \\
We can examine the status of $\pi$ or $\Gamma$ by using the index operator, []. The index itself is context-sensitive: $\pi$ is indexed by some \textit{field}, $\Gamma$ is indexed by one of its substructures. We will use the assignment operator, $\leftarrow$, within the index operator to assign a bitfield to a value or the value of another bitfield.
A field in $\pi$ can be set to a literal value or copied from a bitfield or register. For example:
\begin{equation}
  \pi[\textit{field} \leftarrow \textit{value}]
\end{equation}
\begin{equation}
  \pi[\textit{field} \leftarrow \mathbb{r}]
\end{equation}
\begin{equation}
  \pi[\textit{field} \leftarrow \textit{bitfield}]
\end{equation}
A variable in $\Gamma$ is accessed similarly, but uses the dot operator, ., to access its substructures within the index operator. Example:
\begin{equation}
  \Gamma[\Gamma.\mathct{key} \leftarrow \textit{bitfield}]
\end{equation}
One should not confuse the range operator, [), with the index operator. To avoid ambiguity, the range operator \textit{always} appears inside
of the index operator, []. Example:
\begin{equation}
  \pi[field_1 \leftarrow field_2[0, 10)]
\end{equation}

Comments:
\begin{enumerate}
\item Field \textit{field} with a packet, $\pi$, is set to some literal value, \textit{value}.
\item The value of register $\mathbb{r}$ is copied into \textit{field}.
\item The value of some bitfield, \textit{bitfield} is copied into \textit{field}.
\item The value of the $\mathct{key}$ register in $\Gamma$ is set to the value of \textit{bitfield}
\item The bits of $field_1$ in $\pi$ are set to the bits 0, up to but not including 10, in $field_2$.
\end{enumerate}


\subsection{Operational Semantics 5-Tuple}
Turbak and Gifford define operational semantics for a language as the 5-Tuple, $\langle CF, \Rightarrow, FC, IF, OF \rangle$. CF is the set of all possible configurations that an abstract machine may be in. $\Rightarrow$ is the transition relation, a function that transitions from one configuration to another. FC is the set of all final configurations, the possible ending configurations of the machine. IF is the input function, a function that translates an input into the beginning state of the machine. Finally, OF is the output function, or the transition of the final state into some useful value or domain \cite{Turbak2008}. In Pip, we define the 5-tuple as follows:\\
pip = $\langle IF, \ CF, \ \Rightarrow, \ FC, \ OF \rangle$\\
IF = $\textrm{program} \times \pi \rightarrow CF$ \\
CF = $\alpha \times \Gamma \times \pi \times Lambda$ \\
$\Rightarrow$ = Defined in section ``Transition Relation''. \\
FC = $(\Lambda = \varnothing) \times (\alpha = \varnothing) \times \pi \times \Gamma[egress\_port \neq 0]$ \\
OF = $FC \rightarrow$ \texttt{serialization}

\texttt{serialization} is the output state of Pip: a textual logging of all states that the machine entered.
\subsection{Transition Relation}
\setlength{\mathindent}{0pt}

\begin{mdframed}
\begin{gather*}
  (\mathct{advance}~n \cdot \bar\alpha), \langle \pi, \Lambda, \Gamma \rangle
  \Rightarrow
  (\bar\alpha), \langle \pi, \Lambda, \Gamma[\mathct{header}
  \leftarrow \mathct{header} + n] \rangle
\\
  (\mathct{clear} \cdot \bar\alpha), \langle \pi, \Lambda, \Gamma \rangle
  \Rightarrow
  (\bar\alpha), \langle \pi, \varnothing, \Gamma \rangle
\\
  (\mathct{drop} \cdot \bar\alpha), \langle \pi, \Lambda, \Gamma \rangle
  \Rightarrow
  \varnothing, \langle \pi, \varnothing, \Gamma [\mathct{out} \leftarrow 0] \rangle
\\
  (\mathct{write} \ \textit{action} \cdot \bar\alpha), \langle \pi, \Lambda,
  \Gamma \rangle
  \Rightarrow
  (\bar\alpha), \langle \pi, \Lambda \cdot \textit{action}, \Gamma \rangle
\\
  (\mathct{goto} \ T \cdot \bar\alpha),
  \langle \pi, \Lambda, \Gamma \rangle
  \Rightarrow
  (\bar\alpha), \langle \pi, \Lambda,
  \Gamma[\mathct{current\_table} \leftarrow T] \rangle
\\
  (\mathct{set} \ \textit{field} \ \textit{value} \cdot \bar\alpha), \langle \pi,
  \Lambda, \Gamma \rangle
  \Rightarrow
  (\bar\alpha), \langle \pi', \Lambda, \Gamma' \rangle
\\
  (\mathct{output} \ \mathct{port} \ n \cdot \bar \alpha), \langle \pi, \Lambda,
  \Gamma \rangle
  \Rightarrow
  (\bar\alpha), \langle \pi, \Lambda, \Gamma[\mathct{egress\_port}
  \leftarrow \mathct{port} \ n] \rangle
\\
  (\mathct{copy} \ as_1 \ src \ as_2 \ dst \cdot \bar\alpha), \langle \pi,
  \Lambda, \Gamma \rangle
  \Rightarrow
  (\bar\alpha), \langle \pi', \Lambda, \Gamma' \rangle
\end{gather*}
\end{mdframed}

The $\mathct{advance}$ command advances the header offset by \textit{n} bits.
This can be used by a programmer to adjust the decoding offset relative to a
packet header.
The $\mathct{clear}$ command clears the stored action list. This can be used
by a program reset any accumulated actions if certain conditions are detected
during packet analysis (e.g., filtering flows based on UDP packet content).
The $\mathct{drop}$ command truncates both the action list and stored action list,
and sets the output port to 0 (the null port). 
The effect of this to cause the program to terminate such that the execution 
environment will not forward the packet.
The \mathct{write} command appends an action to the stored action list. This is
the only way to write to the stored action list, allowing for some action to be
taken upon egress, providing the packet is not dropped.
The \mathct{goto} command jumps from the current match table to a new table. The
key register is cleared and rewritten based on the new table's rules. This can
be used as a rudimentary control structure.
The \mathct{set} command sets a bitfield, \textit{field} to some value \textit{value}.
This is used to set fields to a literal value.
The \mathct{output} action outputs to the specified \mathct{port}, \textit{n}, beginning
egress processing.
The \mathct{copy} action copies from one bitfield, \textit{src} to another bitfield, \textit{dst}.
A bitfield can be copied from any address space to any other address space, except for key registers,
which cannot be used as a source.

Definition of copy function:
\begin{mdframed}
\begin{gather*}
  (\mathct{copy}  \ meta \ src \ packet \ dst \ n)
  \rightarrow
  \langle \pi[dst \leftarrow \Gamma.\mathct{meta}[dst, dst + n)],
  \Lambda, \Gamma \rangle
\\
  (\mathct{copy} \ meta \ src \ header \ dst \ n)
  \rightarrow
  \langle \pi[dst + \mathct{hoff} \leftarrow \Gamma.meta[dst, dst + n)],
  \Lambda, \Gamma \rangle
\\
  (\mathct{copy} \ meta \ src \ key \ dst \ n)
  \rightarrow
  \langle \pi, \Lambda, \Gamma[\Gamma.\mathct{key} \leftarrow
  \Gamma.\mathct{meta}[meta, meta + n)] \rangle
\\
  (\mathct{copy} \ packet \ src \ packet \ dst \ n)
  \rightarrow
  \langle \pi[dst \leftarrow \pi[src, src + n)], \Lambda, \Gamma \rangle
\\ 
  (\mathct{copy} \ packet \ src \ header \ dst \ n)
  \rightarrow
  \langle \pi[dst + \mathct{hoff} \leftarrow
  \pi[src, src + n)],
  \Lambda, \Gamma \rangle
\\ 
  (\mathct{copy} \ packet \ src \ key \ dst \ n)
  \rightarrow
  \langle \pi, \Lambda, \Gamma[\Gamma.\mathct{key} \leftarrow \pi[src, src + n)]
  \rangle
\\
  (\mathct{copy} \ header \ src \ header \ dst \ n)
  \rightarrow
  \langle \pi[dst + \mathct{hoff} \leftarrow
  \pi[src + \mathct{hoff}, src + \mathct{hoff} + n)],
  \Lambda, \Gamma \rangle
\\
  (\mathct{copy} \ header \ src \ packet \ dst \ n)
  \rightarrow
  \langle \pi[dst \leftarrow
  \pi[src + \mathct{hoff}, src +
  \mathct{hoff} + n)], \Lambda, \Gamma \rangle
\\
  (\mathct{copy} \ header \ src \ key \ dst \ n)
  \rightarrow
  \langle \pi, \Lambda, \Gamma[\Gamma.\mathct{key}[dst] \leftarrow
  \pi[src + 
  \mathct{hoff}, src + \mathct{hoff} + n)] \rangle
\\
  (\mathct{copy} \ header \ src \ meta \ dst \ n)
  \rightarrow
  \langle \pi, \Lambda, \Gamma[\Gamma.\mathct{meta}[dst] \leftarrow
  \pi[src + 
  \mathct{hoff}, src + \mathct{hoff} + n)] \rangle
\end{gather*}
\end{mdframed}

Defintion of set function:
\begin{mdframed}
\begin{gather*}
  (\mathct{set} \ (meta \ field \ n) \ value)
  \rightarrow
  \langle \pi, \Lambda, \Gamma[\Gamma.\mathct{meta}[field, field + n) \leftarrow value] \rangle
  \\
  (\mathct{set} \ (key \ field \ n) \ value)
  \rightarrow
  \langle \pi, \Lambda, \Gamma[\Gamma.\mathct{key}[field, field + n) \leftarrow value] \rangle
  \\
  (\mathct{set} \ (packet \ field \ n) \ value)
  \rightarrow
  \langle \pi, \Lambda, \Gamma[\Gamma.\mathct{packet}[field, field + n) \leftarrow value] \rangle
  \\
  (\mathct{set} \ (header \ field \ n) \ value)
  \rightarrow
  \langle \pi, \Lambda, \Gamma[\Gamma.\mathct{packet}[field + \mathct{hoff},
      field + \mathct{hoff} + n) \leftarrow value] \rangle
  \\
\end{gather*}
\end{mdframed}
\textit{Note:} the context variable $\mathct{header\_offset}$ is abbreviated to $\mathct{hoff}$ for brevity.
