\documentclass[english,submission]{programming}
\usepackage[fleqn]{amsmath}
\usepackage{amssymb}
%% \usepackage{bbold}
\usepackage{bm}
\usepackage{mathpartir}
\usepackage[framemethod=default]{mdframed}
\usepackage[backend=bibtex, style=ieee]{biblatex}
\bibliography{paper}
\usepackage{algorithm}
\usepackage[noend]{algpseudocode}
\usepackage{hyperref}
\usepackage{pdfpages}
\usepackage[margin=1in]{geometry}
\usepackage[none]{hyphenat}

\newcommand{\mathct}[1]{\bm{\mathsf{{#1}}}}
\newcommand\tab[1][0.5cm]{\hspace*{#1}}

\lstdefinelanguage{pip}
{
  morecomment=[l][\color{black}\textit]{;}
}
\lstset 
{
  language=pip,
  emph={
    pip, table, rules, rule, actions
  },
  emph=[2]{
    copy, match, drop, miss, goto, output, write, advance, set, clear
  },
  emph=[3]{
    int, port, exact, bits, named\_field, exact
  },
  emphstyle={\color{Blue1}\textbf},
  emphstyle=[2]{\color{Purple}},
  emphstyle=[3]{\color{ForestGreen}},
  tabsize=2
}

\begin{document}
\title{Pip: An Abstract Dataplane and Virtual Machine}

\author{Samuel Goodrick}
\author{Andrew Sutton}
\affiliation{The University of Akron}

\paperdetails{
  perspective=sciencetheoretical,
  area={Interpreters, virtual machines, and compilers}
}


\begin{CCSXML}
<ccs2012>
<concept>
<concept_id>10003752.10003766.10003771</concept_id>
<concept_desc>Theory of computation~Grammars and context-free languages</concept_desc>
<concept_significance>300</concept_significance>
</concept>
</ccs2012>
\end{CCSXML}

\ccsdesc[300]{Theory of computation~Grammars and context-free languages}

\maketitle

%% 
\section*{Abstract}
We present an abstract machine and S-expression-based programming language to describe OpenFlow-style software-defined networking. The implemented Pip virtual machine and language provide facilities for packet decoding, safely writing and setting bitfields within packets, and switching based on packet contents. We have outlined an abstract syntax and structural operational semantics for Pip, thus allowing Pip programs to have predictable and provable properties.
Pip allows for easy and safe access and writing to packet fields, as well as a programmable packet pipeline that will rarely stall.

\begin{abstract}
TODO: rewrite this according to http://programming-journal.org/submission/
  
We present an abstract machine and S-expression-based programming language to describe OpenFlow-style software-defined networking. The implemented Pip virtual machine and language provide facilities for packet decoding, safely writing and setting bitfields within packets, and switching based on packet contents. We have outlined an abstract syntax and structural operational semantics for Pip, thus allowing Pip programs to have predictable and provable properties.
Pip allows for easy and safe access and writing to packet fields, as well as a programmable packet pipeline that will rarely stall.
\end{abstract}

\section{Introduction}

%% Answer the following questions:
%% \begin{itemize}
%% \item What is the problem we are solving?
%% \item How are we going to solve the problem?
%% \item What did we do?
%% \item What are the results?
%% \item What are our contributions?
%% \end{itemize}

Conventional switches are configured using vendor-specific interfaces that lack scalability and flexibility. In addition, manual switch configuration is heavily prone to error and compromise. Software-defined networking (SDN) is a paradigm that helps to mitigate user error in packet switching and increase network scalability by allowing switches to be programmed through a high-level programming language. However, there is no standard language for SDN, some programmers have tried to use C or Python, others have developed domain-specific languages to achieve these purposes. There are problems using both general purpose languages and domain-specific languages. The semantics of general purpose languages allow program constructs that allow for invocation of undefined behaviors (buffer overflow, pointer errors, data races, and so on). On the other hand, domain-specific languages, may not be sufficient to express the kinds of computations necessary for higher-level SDN applications. We attempted to build a virtual system for SDN that allows to study and reason about SDN programs, allowing us to identify what properties of a language are most useful for safely and efficiently supporting SDN.

%% Problem: We want either to build a programming language for SDN, or to identify existing languages that might extended to safely and efficiently support SDN.
%% To do this, we need to find a ``sweet spot'' between the two extremes of fully general vs. highly specific languages.

%% \textbf{How did we approach the problem}.
Our approach is to create a minimal abstract machine to represent the kinds of computational facility present in a typical OpenFlow-like data plane.
The description of our abstract machine includes basic notions of state and the smallest possible set of operations needed to define SDN programs.
In essence, we attempted to construct an assembly language for OpenFlow-style packet processing. We defined the abstract syntax of a programming language and the operational semantics for this abstract machine. In order better experiment with it, we also implemented a virtual machine capable of executing SDN programs.

%% \textbf{How does the abstract machine help us study language design?}
Now that we have a model, we hope to start reasoning about higher level language constructs that support safe and efficient switch configurations. We hope to use our findings to construct minimal abstract machines out of subsets of other dataplane domain-specific languages. Operational semantics of a language give a good basis for proof through logic and dynamic logic. In the future, we would like to examine techniques for dynamic logical analysis of dataplane programs.

%% \textbf{What did we do?}

%% \begin{verbatim}
%% Talk about the clean separation of the data plane machine from the controller.
%% The controller has to be a GP language, the DP does not.
%% \end{verbatim}
The Pip language has no object model or declaration system, as well as minimal control structure. By removing much of the freedom present in general-purpose programming languages and restricting the programmer to simple packet-switching actions and buffer modification, Pip programs are able to maintain provable properties and provable termination. The operational semantics of Pip allow for a logical description of the state of the program at any time. In order to maintain these provable properties, we allow the programmer to output a modified packet to a ``controller'', a program written in a general-purpose programming language that has the facilities for mutable state or abstract-syntax tree modification.

%% Built experimental controllers to experiment with mutable program state.

%% \textbf{What did we find out?}
Our abstract machine provides a lot of opportunities for executing undefined behaviors. In particular, it is easy to build programs that read across packet boundaries, fail to validate input, etc. 

Separating the controller from the dataplane allows for packet switching to be reasoned about mathematically without introducing undecidable outcomes.

%% \textbf{Contributions}
%% \begin{enumerate}
%% \item Operational semantics for Pip
%% \item Identify opportunities for undefined behavior
%% \item Implementation of Pip
%% \item Experimental controller programs?s
%% \end{enumerate}

% The problem with typical programming languages is that the semantics of these programming languages are informally designed, often rife with side effects, and perhaps most pertinent for packet switching: prone to buffer overflows. We have presented a new domain specific programming language and virtual machine to model an OpenFlow-style network, with a formally-defined type system and structural operational semantics, thus allowing properties to be proven about software-defined networking programs.

This paper is organized as follows: Section 2 demonstrates the use of the Pip programming language to create simple SDN applications. Section 3 gives a detailed explanation of and serves as a reference to the Pip programming language. It provides proper documentation to the various language constructs of Pip.

Section 4 describes the abstract syntax of the Pip language, first presenting the types and expressions of the language before defining the proper formation of actions, or the instructions of the language. Section 5 explains the operational semantics and state of the machine, and shows a transition relation for all actions presented in Section 4.

Section 6 describes the implementation of the virtual machine and language interpreter.

Section 7 provides discussion of some undefined behaviors and issues with the language. Section 8 discusses future work and solutions to the issues of Pip discovered in related work.


\section{Example programs}

Here we will describe three simple, static, stateless Pip programs: a drop, echo, and static wire program. More complicated (stateful) programs can be created by outputting to the controller port and dynamically adjusting the abstract syntax tree. Doing so removes any proveable properties about a Pip program, and thus will be discussed in a later chapter.

%% \subsection{Drop}
%% The drop program receives a program and outputs it to all physical ports.
%% \begin{verbatim}
%% (pip
%%   (table drop exact
%%     (actions
%%       (copy
%% 	(bitfield physical_port (int i32 0) (int i32 64))
%% 	(bitfield key (int i32 0) (int i32 64))
%% 	(int i32 64))
%%       (match)
%%     ) ; actions
%%     (rules
%%       (rule (miss)
%%         (actions (output (reserved_port all)))
%%       )
%%     )
%%   )
%% )
%% \end{verbatim}
%% This program creates an exact-match table called \textit{drop} and copies the the 32-bit physical port into the 64-bit key register, completely overwriting it if anything was there. 32 0s are padded onto the physical port in this case.

%% Matching begins, wherein there is one rule. The sole rule being a \textit{miss} means its action list will always execute, thus all packets received will be outputted to the \textbf{all} port.
%% \subsection{Echo}
%% Echo simply returns a packet to sender. One can see it is almost identical to the Drop program, except that it outputs to the packet's \textbf{in\_port}, or physical ingress port.
%% \begin{verbatim}
%% (pip
%%   (table echo exact
%%     (actions
%%       (copy
%% 	(bitfield physical_port (int i32 0) (int i32 64))
%% 	(bitfield key (int i32 0) (int i32 64))
%% 	(int i32 64))
%%       (match)
%%     ) ; actions
%%     (rules
%%       (rule (miss)
%%         (actions (output (reserved_port in_port)))
%%       )
%%     )
%%   )
%% )
%% \end{verbatim}

\subsection{Static Wire}
\indent A wire is a common network function involving two ports: when one of the ports receives a packet, it outputs on the other port. In general, the wire is not aware of either port and must learn of their existence dynamically (cite hoang). Pip does not allow for dynamic state changes, thus since the wire does not have any capability to learn of ports dynamically, will simply send a packet to the other of two known ports, port 1 and port 2.
\begin{verbatim}
(pip
  (table wire exact
    (actions
      (copy
	(bits physical_port 0 32)
	(bits key 0 32)
	32)
      (match)
    ) ; actions
    (rules
      (rule (port 1)
        (actions (output (port 2)))
      (rule (port 2)
        (actions (output (port 1)))
      (rule (miss)
        (actions (drop))
      )
    )
  )
)
\end{verbatim}

We begin by defining an \textit{exact match} table called $\mathct{wire}$. An exact match table matches on integers. When the table begins execution, 32 bits are copies from the $\mathct{physical\_port}$ of the packet (a context variable representing the physical ingress port of the packet) into the table's $\mathct{key}$ register. When the $\mathct{match}$ action is reached, table prep ends and matching can begin. Three \textit{match rules} are defined: port 1, port 2, and $\mathct{miss}$. If the value of the \mathct{key} register is equal to one of these rules, the rule's \textit{action list} begins execution. For example, if the $\mathct{key}$ register was equal to 1, then the packet would $\mathct{output}$ to port 2. If the $\mathct{key}$ was not equal to 1 or 2, the $\mathct{miss}$ rule would begin execution, dropping the packet.

\subsection{Stateless Firewall}
Here we define a simple TCP/IPv4 firewall that disables some types of packets from accessing the Internet. If the packet is of type IPv4 with protocol TCP, we will drop any packets to port 80 (HTTP) or port 443 (HTTPS). If the packet is a TCP/IPv4 packet but is not attempting to access ports 80 or 443, we will allow the controller to handle egress processing, as the output port of the packet will still need to be determined.
\begin{verbatim}
(pip
  (table ipv4_check exact
    (actions
      (copy 
        (named_field eth.type 0 16)
        (bits key 0 16))
      (match)
    )
    (rules
      (rule (int i16 0x0800)
        (actions (goto tcp_check)))
      (rule (miss)
        (actions (drop)))
    )
  )

  (table tcp_check exact
    (actions
      (copy 
        (named_field ipv4.type 0 8)
        (bits key 0 8))
      (match)
    )
    (rules
      (rule (int i8 0x06)
        (actions (goto firewall)))
      (rule (miss)
        (actions (drop)))
    )
  )

  (table firewall exact
    (actions
      (copy
        (named_field tcp.src 0 16)
        (bits key 0 16))
      (match)
    )
    (rules
      (rule (int i16 80)
        (actions (drop)))
      (rule (int i16 443)
        (actions (drop)))
      (rule (miss)
        (output (port controller)))
    )
  )
)
\end{verbatim}

Here we see a program taking advantage of the $\mathct{goto}$ action control structure. The ipv4\_check table will match on the \textit{Ethernet ethertype} of the packet. If the ethertype is 0x800, IPv4, we will jump to the tcp\_check table, dropping the packet in any other case. In \textit{tcp\_check}, we match on the \textit{IPv4 protocol} field. If it is equal to 0x06, or TCP, we will jump the final table, \textit{firewall}. Here, we will match the \textit{TCP source port} on ports 80 and 443. If there is a match, we drop the packet, disallowing Internet access. Otherwise, we output to the controller $\mathct{port}$ to begin manual egress processing.

\section{The Pip Language}
This chapter (appendix?) will serve as a reference for the grammar and syntax of the Pip programming language.

Pip is an S-expression-based language. There are 6 basic S-expressions used in the Pip language:
\begin{enumerate}
  \item Pip
  \item Table
  \item Action List
  \item Action
  \item Rule List
  \item Rule
\end{enumerate}

\subsection{Pip S-Expression}
Pip S-Expressions are the expressions that contain the program itself, comparable to the main function in C. Pip expressions can contain any amount of table expressions as parameters. Example:
\begin{mdframed}
\begin{verbatim}
(pip (table ...) (table...))
\end{verbatim}
\end{mdframed}

\subsection{Table S-Expression}
Table S-Expressions define a match table. They contain, in order, a name, a table type (exact match tables are currently the only supported table type), a preparation action list (an action list that sets the key register of the table) ending with a $\mathct{match}$ action, and a rule list that defines the matching rules of the table. For example:
\begin{mdframed}
\begin{verbatim}
(table table_name exact
  (actions
    (set (bits key 0 32) (int i32 0))
    (match)
  )
  (rules
    (rule ...)
  )
)
\end{verbatim}
\end{mdframed}
In this table, an exact-match table called \textit{table\_name} is declared. Its key register is $\mathct{set}$ to the 32-bit integer 0 and matching begins. The rules in the rule list will then be examined and possibly executed. Rules and key registers will be explained later in the chapter.

\subsection{Action Lists and Actions}
Actions are the basic instructions in Pip. Actions Lists are simply sequences of one or more actions. Pip has 9 actions, described as follows: 
\begin{enumerate}
  \item $\mathct{write}$ writes an action the stored action list of the virtual machine. 
  \item $\mathct{clear}$ clears all actions in the stored action list.
  \item $\mathct{drop}$ drops the packet and discontinues processing.
  \item $\mathct{match}$ ends the prep list of a table and begins matching.
  \item $\mathct{goto}$ jumps to another table. Only tables that appear later in the program may be jumped to.
  \item $\mathct{output}$ sends the packet out a specific port.
  \item $\mathct{advance}$ increments the context variable $\mathct{header\_offset}$ by the specified amount. This determines the distance in bits of the $\mathct{header}$ address space from the $\mathct{packet}$ address space.
  \item $\mathct{copy}$ copies n bits from one bitfield to another. The programmer may not copy past th eend of an address space.
  \item $\mathct{set}$ sets a bitfield to some literal value.
\end{enumerate}
An example action list follows:
\begin{mdframed}
\begin{verbatim}
(actions
  (set (named_field tcp.dst) 443) (drop))
\end{verbatim}
\end{mdframed}
An action is marked by the $\mathct{action}$ keyword. Individual actions are not marked by a keyword as they only appear within action lists and write action parameters.

\subsection{Rule Lists and Rules}
Rules are the entries in a table that are matched upon. A rule has a key (not to be confused with a $\mathct{key}$ register) that is equality compared to the $\mathct{key}$ register. In an exact match table, a rule key is of type $\mathct{int}(n)$. Along with a key, a rule contains an action list that will be executed if the rule key matches the key register. A rule list is a sequence of one or more rules. An example rule list follows:
\begin{mdframed}
\begin{verbatim}
(rules
  (rule (int i32)
    (actions ...)))
\end{verbatim}
\end{mdframed}
A rule is marked by the $\mathct{rule}$ keyword and a rule list by the $\mathct{rules}$ keyword.

\subsection{Key Register}
The $\mathct{key}$ register is a 64-bit field that is contained in the table data structure. The $\mathct{key}$ register is equality compared to the table's $\mathct{rule}$ keys. In other words, matching in a table works similarly to switch statements in C.

\subsection{Address Spaces}
There are 4 address spaces in the Pip language and virtual machine:
\begin{enumerate}
  \item $\mathct{packet}$ denotes the 0th bit in the packet.
  \item $\mathct{header}$ denotes the 0th bit in the packet plus the value of the $\mathct{header\_offset}$ context variable.
  \item $\mathct{meta}$ is the 64-bit metadata register. It can be used to store values for scratch processing.
  \item $\mathct{key}$ is the 64-bit key register.
\end{enumerate}

\subsection{Reserved Ports}
Pip inherits the following reserved ports from the OpenFlow standard:
\begin{enumerate}
  \item $\mathct{drop}$ or port 0. Outputting to this port has the same effect as the $\mathct{drop}$ action.
  \item $\mathct{all}$ outputs to all ports on the device.
  \item $\mathct{controller}$ sends the packet to some non-Pip program. Our implementation uses C++ programs.
  \item $\mathct{in\_port}$ outputs to the physical ingress port of the packet.
  \item $\mathct{any}$ outputs to a random port.
  \item $\mathct{local}$ outputs to the local IP stack.
\end{enumerate}

\subsection{Named Fields}
Named fields are bitfields; they have type $\mathct{bits}$. They are equivalent in every way to a bitfield in the $\mathct{packet}$ address space, and only exist as syntactic sugar, but help to prevent memory access violations. A named field should always be preferred to a raw bitfield if available. Named fields exist in Pip for Ethernet frames, IPv4 fields, and TCP fields. The naming scheme uses common abbreviations, for example $\mathct{eth.dst}$ representing the Ethernet destination MAC address, and $\mathct{ipv4.len}$ representing the IPv4 header length. IPv6 and UDP will be added in the future.
%% \begin{enumerate}
%%   \item $\mathct{eth.dst}$ 
%%   \item $\mathct{eth.src}$
%%   \item $\mathct{eth.type}$
%%   \item $\mathct{ipv4.vhl}$
%%   \item $\mathct{ipv4.tos}$
%%   \item $\mathct{ipv4.len}$
%%   \item $\mathct{ipv4.id}$
%%   \item $\mathct{ipv4.frag\_offset}$
%%   \item $\mathct{ipv4.ttl}$
%%   \item $\mathct{ipv4.protocol}$
%%   \item $\mathct{ipv4.checksum}$
%%   \item $\mathct{ipv4.src}$
%%   \item $\mathct{ipv4.dst}$
%%   \item $\mathct{tcp.src}$
%%   \item $\mathct{tcp.dst}$
%%   \item $\mathct{tcp.seq}$
%%   \item $\mathct{tcp.ack}$
%%   \item $\mathct{tcp.offset}$
%%   \item $\mathct{tcp.flags}$
%%   \item $\mathct{tcp.window}$
%%   \item $\mathct{tcp.checksum}$
%%   \item $\mathct{tcp.urgent\_ptr}$
%% \end{enumerate}


\section{Abstract syntax}
In this section, we explain the abstract syntax of the Pip programming
language. 
Pip uses a type system to ensure semi-semantically correct instructions. Instructions, known as actions, can take well-formed expressions as parameters, 

\subsection{Types}

\begin{align*}
\tau ::=~& \mathct{int}(n)        & \text{integer types} \\
         & \mathct{bits}(a, o, l) & \text{bitfields} \\
         & \mathct{port}          & \text{ports} \\
         & \mathct{table}         & \text{tables} \\
\end{align*}

\begin{enumerate}
\item $\mathct{int}(n)$ types are unsigned integers with a bit-width of size \textit{n}.
\item $\mathct{bits}$ types are bitfields within the storage space of the virtual machine. Bitfields are defined as having an \textit{address space} that denote which area of storage they reside: packet, header, key, meta, ingress\_port, and physical\_port. A bitfield must also specify its length in bits as well as its offset from bit 0 of the address space, again in bits.
\item $\mathct{port}$ types represent numbered ports. Some reserved ports are named, and occupy a static number. For example, the $\mathct{drop}$ port is assigned to 0.
\item $\mathct{table}$ types are data structures that contain all other Pip constructs. A table has a \textit{key register} to be matched against. The programmer defines the value of the key register, which is computed at runtime. Tables have \textit{rules} which, if equal to the key register, will execute a block of actions.
\end{enumerate}

\subsection{Expressions}
\begin{align*}
  \text{expressions} ::=~& \mathct{int}(w) n    & \text{where} \ w \ \text{and} \ n \ \text{are integer literals.}\\
  & | \ \mathct{port} \ n   & \text{where } n \text{ is an integer literal.}\\
  & | \ \mathct{bits} \ as \ pos \ len &\text{where } pos \text{ and } len \text{ are integer literals.} \\ & &\text{and } as \text{ is an address space.}\\
  & | \ \mathct{miss}\\
  & | \ \mathct{ref} \ id    &\text{where } id \text{ is the name of a table.}
\end{align*}
\subsection{Actions}

Actions are instructions that either have no parameters, an expression as a parameter, or another action as a parameter.
\begin{align*}
actions::=~&  \mathct{write} \ action     & action \text{ must be an } \mathct{action}.\\
  & | \ \mathct{clear}\\
  & | \ \mathct{drop}\\
  & | \ \mathct{match}\\
  & | \ \mathct{goto} \ table             & \text{table must be of type } \mathct{table}.\\
  & | \ \mathct{output} \ port            & \text{port must be of type } \mathct{port}.\\
  & | \ \mathct{advance} \ n              & \text{n must be an integer literal.}\\
  & | \ \mathct{copy} src \ dst \ n       & src \text{ and } dst \text{ must be of type } \mathct{bits} \\
                                         && \text{n must be of type } \mathct{int}(w).\\
  & | \ \mathct{set} field \ value        & field \text{ must be of type } \mathct{bits} \text{ and }\\
                                         && value \text{ must be of type } \mathct{int}(w).
\end{align*}


\section{Operational Semantics}

\subsection{Notation}

The state of the evaluator, $\sigma$, is the tuple consisting of \\
packet, $\pi$\\
action\_list, $\alpha$ \\
stored\_action\_list, $\Lambda$ \\
context, $\Gamma$, which is the tuple: \\
$\langle ingress\_port, \ egress\_port, \ physical\_port, \ key, \ meta, \ header\_offset, \ table \rangle$ \\ \\
Compactly, $\sigma = \langle \pi, \alpha, \Lambda, \Gamma \rangle$
\\ \\
We can examine the status of $\pi$ or $\Gamma$ by using the index operator, []. The index itself is context-sensitive: $\pi$ is indexed by some \textit{field}, $\Gamma$ is indexed by one of its substructures. We will use the assignment operator, $\leftarrow$, within the index operator to assign a bitfield to a value or the value of another bitfield.
A field in $\pi$ can be set to a literal value or copied from a bitfield or register. For example:
\begin{equation}
  \pi[\textit{field} \leftarrow \textit{value}]
\end{equation}
\begin{equation}
  \pi[\textit{field} \leftarrow \mathbb{r}]
\end{equation}
\begin{equation}
  \pi[\textit{field} \leftarrow \textit{bitfield}]
\end{equation}
A variable in $\Gamma$ is accessed similarly, but uses the dot operator, ., to access its substructures within the index operator. Example:
\begin{equation}
  \Gamma[\Gamma.\mathct{key} \leftarrow \textit{bitfield}]
\end{equation}
One should not confuse the range operator, [), with the index operator. To avoid ambiguity, the range operator \textit{always} appears inside
of the index operator, []. Example:
\begin{equation}
  \pi[field_1 \leftarrow field_2[0, 10)]
\end{equation}

Comments:
\begin{enumerate}
\item Field \textit{field} with a packet, $\pi$, is set to some literal value, \textit{value}.
\item The value of register $\mathbb{r}$ is copied into \textit{field}.
\item The value of some bitfield, \textit{bitfield} is copied into \textit{field}.
\item The value of the $\mathct{key}$ register in $\Gamma$ is set to the value of \textit{bitfield}
\item The bits of $field_1$ in $\pi$ are set to the bits 0, up to but not including 10, in $field_2$.
\end{enumerate}


\subsection{Operational Semantics 5-Tuple}
Turbak and Gifford define operational semantics for a language as the 5-Tuple, $\langle CF, \Rightarrow, FC, IF, OF \rangle$. CF is the set of all possible configurations that an abstract machine may be in. $\Rightarrow$ is the transition relation, a function that transitions from one configuration to another. FC is the set of all final configurations, the possible ending configurations of the machine. IF is the input function, a function that translates an input into the beginning state of the machine. Finally, OF is the output function, or the transition of the final state into some useful value or domain.
pip = $\langle IF, \ CF, \ \Rightarrow, \ FC, \ OF \rangle$\\
IF = $\textrm{program} \times \pi \rightarrow CF$ \\
CF = $\alpha \times \Gamma \times \pi \times Lambda$ \\
$\Rightarrow$ = Defined in section ``Transition Relation''. \\
FC = $(\Lambda = \varnothing) \times (\alpha = \varnothing) \times \pi \times \Gamma[egress\_port \neq 0]$ \\
OF = $FC \rightarrow$ \texttt{serialization}

\texttt{serialization} is the output state of Pip: a textual logging of all states that the machine entered.
\subsection{Transition Relation}
\setlength{\mathindent}{0pt}

\begin{mdframed}
\begin{gather*}
  (\mathct{advance}~n \cdot \bar\alpha), \langle \pi, \Lambda, \Gamma \rangle
  \Rightarrow
  (\bar\alpha), \langle \pi, \Lambda, \Gamma[\mathct{header}
  \leftarrow \mathct{header} + n] \rangle
\\
  (\mathct{clear} \cdot \bar\alpha), \langle \pi, \Lambda, \Gamma \rangle
  \Rightarrow
  (\bar\alpha), \langle \pi, \varnothing, \Gamma \rangle
\\
  (\mathct{drop} \cdot \bar\alpha), \langle \pi, \Lambda, \Gamma \rangle
  \Rightarrow
  \varnothing, \langle \pi, \varnothing, \Gamma [\mathct{out} \leftarrow 0] \rangle
\\
  (\mathct{write} \ \textit{action} \cdot \bar\alpha), \langle \pi, \Lambda,
  \Gamma \rangle
  \Rightarrow
  (\bar\alpha), \langle \pi, \Lambda \cdot \textit{action}, \Gamma \rangle
\\
  (\mathct{goto} \ T \cdot \bar\alpha),
  \langle \pi, \Lambda, \Gamma \rangle
  \Rightarrow
  (\bar\alpha), \langle \pi, \Lambda,
  \Gamma[\mathct{current\_table} \leftarrow T] \rangle
\\
  (\mathct{set} \ \textit{field} \ \textit{value} \cdot \bar\alpha), \langle \pi,
  \Lambda, \Gamma \rangle
  \Rightarrow
  (\bar\alpha), \langle \pi', \Lambda, \Gamma' \rangle
\\
  (\mathct{output} \ \mathct{port} \ n \cdot \bar \alpha), \langle \pi, \Lambda,
  \Gamma \rangle
  \Rightarrow
  (\bar\alpha), \langle \pi, \Lambda, \Gamma[\mathct{egress\_port}
  \leftarrow \mathct{port} \ n] \rangle
\\
  (\mathct{copy} \ as_1 \ src \ as_2 \ dst \cdot \bar\alpha), \langle \pi,
  \Lambda, \Gamma \rangle
  \Rightarrow
  (\bar\alpha), \langle \pi', \Lambda, \Gamma' \rangle
\end{gather*}
\end{mdframed}

The $\mathct{advance}$ command advances the header offset by \textit{n} bits.
This can be used by a programmer to adjust the decoding offset relative to a
packet header.
The $\mathct{clear}$ command clears the stored action list. This can be used
by a program reset any accumulated actions if certain conditions are detected
during packet analysis (e.g., filtering flows based on UDP packet content).
The $\mathct{drop}$ command truncates both the action list and stored action list,
and sets the output port to 0 (the null port). 
The effect of this to cause the program to terminate such that the execution 
environment will not forward the packet.
The \mathct{write} command appends an action to the stored action list. This is
the only way to write to the stored action list, allowing for some action to be
taken upon egress, providing the packet is not dropped.
The \mathct{goto} command jumps from the current match table to a new table. The
key register is cleared and rewritten based on the new table's rules. This can
be used as a rudimentary control structure.
The \mathct{set} command sets a bitfield, \textit{field} to some value \textit{value}.
This is used to set fields to a literal value.
The \mathct{output} action outputs to the specified \mathct{port}, \textit{n}, beginning
egress processing.
The \mathct{copy} action copies from one bitfield, \textit{src} to another bitfield, \textit{dst}.
A bitfield can be copied from any address space to any other address space, except for key registers,
which cannot be used as a source.

Definition of copy function:
\begin{mdframed}
\begin{gather*}
  (\mathct{copy}  \ meta \ src \ packet \ dst \ n)
  \rightarrow
  \langle \pi[dst \leftarrow \Gamma.\mathct{meta}[dst, dst + n)],
  \Lambda, \Gamma \rangle
\\
  (\mathct{copy} \ meta \ src \ header \ dst \ n)
  \rightarrow
  \langle \pi[dst + \mathct{hoff} \leftarrow \Gamma.meta[dst, dst + n)],
  \Lambda, \Gamma \rangle
\\
  (\mathct{copy} \ meta \ src \ key \ dst \ n)
  \rightarrow
  \langle \pi, \Lambda, \Gamma[\Gamma.\mathct{key} \leftarrow
  \Gamma.\mathct{meta}[meta, meta + n)] \rangle
\\
  (\mathct{copy} \ packet \ src \ packet \ dst \ n)
  \rightarrow
  \langle \pi[dst \leftarrow \pi[src, src + n)], \Lambda, \Gamma \rangle
\\ 
  (\mathct{copy} \ packet \ src \ header \ dst \ n)
  \rightarrow
  \langle \pi[dst + \mathct{hoff} \leftarrow
  \pi[src, src + n)],
  \Lambda, \Gamma \rangle
\\ 
  (\mathct{copy} \ packet \ src \ key \ dst \ n)
  \rightarrow
  \langle \pi, \Lambda, \Gamma[\Gamma.\mathct{key} \leftarrow \pi[src, src + n)]
  \rangle
\\
  (\mathct{copy} \ header \ src \ header \ dst \ n)
  \rightarrow
  \langle \pi[dst + \mathct{hoff} \leftarrow
  \pi[src + \mathct{hoff}, src + \mathct{hoff} + n)],
  \Lambda, \Gamma \rangle
\\
  (\mathct{copy} \ header \ src \ packet \ dst \ n)
  \rightarrow
  \langle \pi[dst \leftarrow
  \pi[src + \mathct{hoff}, src +
  \mathct{hoff} + n)], \Lambda, \Gamma \rangle
\\
  (\mathct{copy} \ header \ src \ key \ dst \ n)
  \rightarrow
  \langle \pi, \Lambda, \Gamma[\Gamma.\mathct{key}[dst] \leftarrow
  \pi[src + 
  \mathct{hoff}, src + \mathct{hoff} + n)] \rangle
\\
  (\mathct{copy} \ header \ src \ meta \ dst \ n)
  \rightarrow
  \langle \pi, \Lambda, \Gamma[\Gamma.\mathct{meta}[dst] \leftarrow
  \pi[src + 
  \mathct{hoff}, src + \mathct{hoff} + n)] \rangle
\end{gather*}
\end{mdframed}

Defintion of set function:
\begin{mdframed}
\begin{gather*}
  (\mathct{set} \ (meta \ field \ n) \ value)
  \rightarrow
  \langle \pi, \Lambda, \Gamma[\Gamma.\mathct{meta}[field, field + n) \leftarrow value] \rangle
  \\
  (\mathct{set} \ (key \ field \ n) \ value)
  \rightarrow
  \langle \pi, \Lambda, \Gamma[\Gamma.\mathct{key}[field, field + n) \leftarrow value] \rangle
  \\
  (\mathct{set} \ (packet \ field \ n) \ value)
  \rightarrow
  \langle \pi, \Lambda, \Gamma[\Gamma.\mathct{packet}[field, field + n) \leftarrow value] \rangle
  \\
  (\mathct{set} \ (header \ field \ n) \ value)
  \rightarrow
  \langle \pi, \Lambda, \Gamma[\Gamma.\mathct{packet}[field + \mathct{hoff},
      field + \mathct{hoff} + n) \leftarrow value] \rangle
  \\
\end{gather*}
\end{mdframed}
\textit{Note:} the context variable $\mathct{header\_offset}$ is abbreviated to $\mathct{hoff}$ for brevity.

\section{Implementation}

Pip was implemented in C++ using Dr. Andrew Sutton's Sexpr and CC libraries. Interpretation of a Pip program happens in five stages: input, sexpr-parsing, translation, name resolution and evaluation. Input and sexpr-parsing are handled by libsexpr and libCC and will be ignored in this paper. The implementation of translation, resolution, and evaluation will be outlined in this chapter. The implementation is hosted on \href{https://github.com/asutton/pip/}{GitHub}.

\subsection{Translation}
The translator takes, as a parameter, a libsexpr expression and returns a declaration. It works as a recursive-descent, syntax-directed translator built around the \texttt{match\_list} framework of libsexpr. The \texttt{match\_list} framework allows for pattern-matching of user-defined s-expressions, with each unique type being parsed by a callback function.

The types that the translator parses are defined below. Keywords and syntax appear in $\mathct{san serif}$ font.
\begin{mdframed}
\begin{align*} %% decl
  \text{d} \in \text{decl} ::=~& \mathct{(table} \ \text{table\_id table\_kind action\_seq rule\_seq}\mathct{)}
\end{align*}
\begin{align*} %% decl_seq
  \text{decl\_seq} ::=~& \text{decl*}
\end{align*}
\begin{align*} %% action
  \text{a} \in \text{action} ::=~& \mathct{(}\text{action\_name}\mathct{)} \\
  & | \ \mathct{(}\text{action\_name expr*}\mathct{)} \\ 
  & | \ \mathct{(}\text{action\_name action*}\mathct{)}
\end{align*}
\begin{align*} %% action_seq
  \text{action\_seq} ::=~& \text{action*}
\end{align*}
\begin{align*} %% rule
  \text{r} \in \text{rule} ::=~& \mathct{(}\text{expr action\_seq}\mathct{)}
\end{align*}
\begin{align*} %% rule_seq
  \text{rule\_seq} ::=~& \text{rule*}
\end{align*}
\begin{align*} %% expr
  \text{e} \in \text{expr} ::=~& \text{int\_expr} \\
  & | \ \text{port\_expr} \\
  & | \ \text{miss\_expr} \\
  & | \ \text{ref\_expr} \\
  & | \ \text{bits\_expr} \\
  & | \ \text{named\_field\_expr}
\end{align*}
\end{mdframed}
Here $\mathct{action\_name}$ refers to the syntactic name of an action as defined in section \textbf{???}.

We can see a faithful implementation of this backus-naur form definition is incredibly simple in C++ and \texttt{match\_list}. Below is the implementation of \texttt{decl\_seq} and \texttt{decl}:
\begin{mdframed}
\begin{verbatim}
decl_seq
translator::trans_decls(const sexpr::expr* e)
{
  decl_seq decls; //decl_seq = std::vector<decl*>
  for(auto el : e) {
    decl* d = trans_decl(e);
    decls.push_back(d);
  }
   
  return decls;
}
  
/// decl ::= table-decl
decl*
translator::trans_decl(const sexpr::expr* e)
{
  symbol* sym; // symbol = std::string*
  match_list(e, &sym);
  if (*sym == "table")
    return trans_table(e);
}
\end{verbatim}
\end{mdframed}
Translation of all types works similarly: a sequence is translated by iterating through each s-expression within a list of s-expressions. The parser gets recursively translates each element of the s-expression into a Pip AST, storing the completed ASTs in a vector. We see in this example, an overloading of \texttt{match\_list} to handle symbols, or string pointers, although the function can take any amount of arguments. Defining overloads of \texttt{match\_list} functions is beyond the scope of this paper.
           
\subsection{Resolution}
Name resolution is responsible for performing name lookup after the first translation phase. It is responsible for assigning meanings to $\mathct{ref\_expr}$, or references to table names.

Resolution itself has two phases: first, a table linking locations in memory to identifiers is created; second, those identifiers are replaced with their pointers to their location.
\subsection{Evaluation}
The final stage of interpretation is evaluation. The evaluator framework is an entire virtual machine, containing the entire Pip state: a context, packet, action list, and stored action list. An evaluator operates upon a single packet; a Pip program that uses $n$ packets will have $n$ evaluators.

Upon construction of an evaluator, all table declarations are initialized with their static rule lists. The first table in the program is set as the $\mathct{current\_table}$ and its prep-action-list is queued into the evaluator. The evaluator will execute each action in its queue until the queue is empty. An abridged pseudocode version of evaluator construction is defined below:
\begin{algorithm*}
\caption{Evaluator Construction}
$\mathct{let}$ tables = vector of declarations \\
$\mathct{foreach}$ table $\in$ program \\
\tab $\mathct{insert}$ table into tables \\
\tab $\mathct{foreach}$ rule $\in$ table \\
\tab\tab $\mathct{insert}$ rule.key into hashtable \\
\end{algorithm*}

The main algorithm of the evaluator is the \texttt{step} algorithm, which switches on the next action in the queue, and recursively executes them in turn.
\begin{algorithm}
\caption{step}
\begin{algorithmic}
  \While{!empty(eval)}
  \State action a $\leftarrow$ front(eval)
  \If{get\_kind(a) = advance\_action}
    \Return eval\_advance(a)
  \ElsIf{get\_kind(a) = copy\_action}
    \Return eval\_copy(a)
  \EndIf
  \State ...
  \EndWhile
\end{algorithmic}
\end{algorithm}

The most complex algorithms are found in the evaluation of actions. The evaluation of the copy action, for example, involves copying bits in place from integer values into byte arrays, odd numbers of bites from one byte array to another, and from byte arrays back into integers. These become incredibly dense very fast and will not be examined. In the name of brevity, we will only study the most important action, $\mathct{match}$. The \texttt{eval\_match} algorithm looks up the key register's value in a hashtable of the table's rule keys. If there is a match, the current rule's action list will be added to the evaluator queue. The hashtable find function returns the integer identifier of the rule, so that the matched rule can be kept track of. If there is no match, then the miss rule's action list will be enqueued. If a miss rule is not defined and the packet is not matched, the packet will be dropped implicitly.
\begin{algorithm}
\caption{eval\_match}
$\mathct{for}$ rule $\in$ current\_table.rules\\
\tab key $\leftarrow$ rule.key\\
\tab id $\leftarrow$ find(hashtable, key)\\
\tab $\mathct{if}$(id $\neq$ 0)\\
\tab\tab insert(eval, $\textrm{current\_table.rules}_{id}$.action\_list)\\
insert(eval, $\textrm{current\_table.rules}_{miss}$.action\_list
\end{algorithm}

Eventually no more actions will be left to evaluate. When this happens, the stored action list will get its chance to execute. The process is exactly the same.

\section{Discussion}
In this chapter, we will examine some undefined behaviors and their mitigations if any.

\subsection{Goto}
The most glaring case of undefined behavior is the ``infinite loop'', caused by a two tables jumping back and forth.
%% \begin{mdframed}
\begin{lstlisting}
(pip
  (table error1 exact
    (actions
      (set (bits key 0 64))
      (match)
    ) ; actions
    (rules
      (rule (miss)
        (actions (goto (error2)))
      )
    )
  )
  (table error2 exact
    (actions
      (set (bits key 0 64))
      (match)
    ) ; actions
    (rules
      (rule (miss)
        (actions (goto (error1)))
      )
    )
  )
)
\end{lstlisting}
%% \end{mdframed}
This is an invalid program in Pip. All tables have sequential integer identifiers, and in order for a $\mathct{goto}$ action to be valid, the table being jumped to must have a higher integer identifier than the current table. In other words, the table being jumped to must come later in the program.

\subsection{Terminators}
The actions $\mathct{drop}$, $\mathct{output}$, and $\mathct{match}$ are \textit{terminator} actions, meaning they can only come as the last action in an action list. All action lists end in a terminator. Thus the following undefined-behavior-invoking action list is ill-formed.
%% \begin{mdframed}
\begin{lstlisting}
(actions (output (reserved_port controller)) 
         (output (reserved_port in_port)))
\end{lstlisting}
%% \end{mdframed}
The $\mathct{match}$ terminator action is slightly different than the others. All exact-match tables \textit{must} begin their matching phase, even if no rules are defined. The $\mathct{match}$ action \textit{must} appear at the end of the table's prep-list, or action list that sets the key. Thus, the following program is ill-formed.
%% \begin{mdframed}
\begin{lstlisting}
(pip
  (table error exact
    (actions
      (set (bits key 0 64))
    ) ; actions
    (rules
      (rule (miss)
        (actions (output (reserved_port controller)))
      )
    )
  )
)
\end{lstlisting}
%% \end{mdframed}
As is a program that matches outside of the prep-list.
%% \begin{mdframed}
\begin{lstlisting}
(rule (miss)
  (actions (clear) (match)))
\end{lstlisting}
%% \end{mdframed}

\subsection{Overlapping Writes}
Due to Pip's bitwise, storage-based memory model, there is no mechanism preventing a programmer from writing over only part of a buffer that was already written. Although this could possibly be useful for some advanced bit-manipulation algorithms, more often than not, reading this memory will invoke undefined behavior.
%% \begin{mdframed}
\begin{lstlisting}
(actions (copy (bits meta 0 48)
               (eth.dst)
               48)
         (copy (bits packet 48 16)
               (bits packet 0 16)
               16))
\end{lstlisting}
%% \end{mdframed}
Here, the Ethernet destination mac address (the first 48 bits of \texttt{packet}) is written to in full by the first copy action. The second copy action writes only the first 16 bits of the same header, resulting in an unknown value. Pip does not currently mitigate this error.


\section{Future Work}
Pip has a type system that ensures instructions can be written correctly, but does not guarantee that they produce meaningful results. For example, the programmer can copy from anywhere in the buffer to anywhere else in the buffer. Using named bitfields gives the programmer a safe alternative to bit manipulation, but the programmer is not restricted to them. Pip's storage-based model means that it lacks an object model or declaration system (beyond declaring tables). A declaration system would mitigate, and possibly eliminate memory access violations at the expense of freedom. The similar dataplane programming language, Steve, attempted to create a declaration system for packet switching.

Steve's declaration system was described in a paper by H. Nguyen \cite{Nguyen2016}. Steve has a user-defined type known as a \textit{layout} that allows programmers to dynamically define named bitfields within the packet. An example Steve layout follows:
\begin{mdframed}
\begin{verbatim}
layout ethernet {
  dst : uint(48);
  src : uint(48);
  type : uint(16);
}
\end{verbatim}
\end{mdframed}
Once a layout has been defined, the programmer can \textit{extract} the individual headers and use them like variables.
\begin{mdframed}
\begin{verbatim}
extract ethernet.type;

if(eth.type == 0x600) {
 // ...
}
\end{verbatim}
\end{mdframed}
Until the programmer has explicitly \textit{extracted} a named field, it is not usable. Thus, the only part of the buffer accessible to the programmer are the fields they have explicitly defined and stated they wish to access. Barring poor definition, there is no possible way for the programmer to overwrite their working buffer or access beyond its boundary \cite{Nguyen2016}. Defining a similar declaration system in Pip would be a step toward fixing this issue.

As stated before, outputting to a controller program removes any provability of a Pip program. To help programmers reason about dynamic programs, we wish to create a debugger.



\section{Literature Survey}
1) What is the problem they try to solve?
2) How are they solving it?
3) What is the difference between their research and ours?  (What they solve vs. what we solve)

\textbf{Languages for Software-Defined Networks} 

The Frenetic project aimed to raise the level of abstraction for 
programming SDNs \cite{Foster2013}. Frenetic provides a suite of declarative abstractions 
for querying network state, defining forwarding policies, and updating 
policies in a consistent way. This allows the project to replace the 
available low-level imperative interfaces. This was attained by designing 
the constructs in a modular way and implementing them by compiling
them down to low-level OpenFlow forwarding rules. \\

\noindent \textbf{Eliminating Network Protocol Vulnerabilities Through Abstraction and Systems Language Design
}

In a paper written by Casey, a systems programming language with abstractions capturing network
protocol vulnerability constraints was introduced. \cite{Casey2013}. This programming language
allows programmers to capture network protocol message structure and constraints. 


\noindent \textbf{Design Principles for Packet Parsers}


\cite{Gibb2013} \\

\noindent \textbf{Extending Networking into the Virtualization Layer}

A network switch built specifically for virtual environments, Open vSwitch, exports an external interface
for fine-grained control of configuration state and forwarding behavior. Open vSwitch provides
connectivity between virtual machines and physical interfaces. It exports interfaces for manipulating
the forwarding state and managing configuration state at runtime  
\cite{Pfaff2009} \\



Huong Thesis - solve a much larger problem (the citations for him and Michael's thesis are in master)

Michael's thesis - implemented a VM that would run steve programs (don't have to say much about that, it was a full VM more so than ours)



% Prints References section
%\printbibliography

\printbibliography[title=Bibliography]


\end{document}
